\documentclass[
	% -- opções da classe memoir --
	article,			% indica que é um artigo acadêmico
	11pt,				% tamanho da fonte
	oneside,			% para impressão apenas no recto. Oposto a twoside
	a4paper,			% tamanho do papel. 
	% -- opções da classe abntex2 --
	%chapter=TITLE,		% títulos de capítulos convertidos em letras maiúsculas
	%section=TITLE,		% títulos de seções convertidos em letras maiúsculas
	%subsection=TITLE,	% títulos de subseções convertidos em letras maiúsculas
	%subsubsection=TITLE % títulos de subsubseções convertidos em letras maiúsculas
	% -- opções do pacote babel --
	english,			% idioma adicional para hifenização
	brazil,				% o último idioma é o principal do documento
	sumario=tradicional
	]{abntex2}

% ---
% Pacotes básicos 
% ---
\usepackage[T1]{fontenc}		% Selecao de codigos de fonte.
\usepackage[utf8]{inputenc}		% Codificacao do documento (conversão automática dos acentos)
\usepackage{helvet}		% fira sans :/			
\usepackage{lastpage}			% Usado pela Ficha catalográfica
\usepackage{indentfirst}		% Indenta o primeiro parágrafo de cada seção.
\usepackage{nomencl} 			% Lista de simbolos
\usepackage{color}				% Controle das cores
\usepackage{graphicx}			% Inclusão de gráficos
\usepackage{microtype} 	 			% para melhorias de justificação

\renewcommand{\familydefault}{\sfdefault}
% ---
% Configurar ambiente tabular para usar fonte serifada
\let\oldtabular\tabular
\let\endoldtabular\endtabular
\renewenvironment{tabular}{\rmfamily\oldtabular}{\endoldtabular}

% Configurar tabularx se estiver usando
\usepackage{tabularx}
\renewcommand{\tabularxcolumn}[1]{>{\rmfamily}m{#1}}

% Manter legendas de tabelas em serifada
\usepackage{caption}
\captionsetup[table]{font=rm, labelfont=bf}

% Manter serifada em ambientes matemáticos nas tabelas
\usepackage{etoolbox}
\AtBeginEnvironment{tabular}{\rmfamily}
\AtBeginEnvironment{tabularx}{\rmfamily}
\AtBeginEnvironment{longtable}{\rmfamily}
% ---
% Pacotes adicionais, usados apenas no âmbito do Modelo Canônico do abnteX2
% ---
\usepackage{lipsum}				% para geração de dummy text
% ---

% ---
% Pacotes de citações
% ---
\usepackage[brazilian,hyperpageref]{backref}	 % Paginas com as citações na bibl
\usepackage[alf]{abntex2cite}	% Citações padrão ABNT

% --- 
% CONFIGURAÇÕES DE PACOTES
% --- 

% ---
% Configurações do pacote backref
% Usado sem a opção hyperpageref de backref
%\renewcommand{\backrefpagesname}{Citado na(s) página(s):~}
% Texto padrão antes do número das páginas
%\renewcommand{\backref}{}
% Define os textos da citação
%\renewcommand*{\backrefalt}[4]{
%	\ifcase #1 %
%		Nenhuma citação no texto.%
%	\or
%		Citado na página #2.%
%	\else
%		Citado #1 vezes nas páginas #2.%
%	\fi}%
%
% Informações de dados para CAPA e FOLHA DE ROSTO
% ---
\titulo{Análise Estatística do Desempenho da Ação do Banco do Brasil (BRAS3) no período de 2004 a 2024}
\tituloestrangeiro{***}
\autor{Gabrieli Soares\and{Leonardo Monteiro}\and{Marina Caponera Silva}}
\instituicao{Universidade Federal de Santa Catarina}
\local{Brasil}
\data{Novembro, 2025}
% ---


% ---
% Configurações de aparência do PDF final

% alterando o aspecto da cor azul
\definecolor{blue}{RGB}{41,5,195}

% informações do PDF
\makeatletter
\hypersetup{
     	%pagebackref=true,
		pdftitle={\@title}, 
		pdfauthor={\@author},
	    pdfcreator={LaTeX with abnTeX2},
		pdfkeywords={abnt}{latex}{abntex}{abntex2}{trabalho acadêmico}, 
		colorlinks=true,       		% false: boxed links; true: colored links
    	    linkcolor=blue,          	% color of internal links
    	    citecolor=blue,        		% color of links to bibliography
    	    filecolor=magenta,      		% color of file links
		urlcolor=blue,
		bookmarksdepth=4
}
\makeatother
% --- 
% compila o indice
% ---
\makeindex
% ---
% Altera as margens padrões
% ---
\setlrmarginsandblock{3cm}{3cm}{*}
\setulmarginsandblock{3cm}{3cm}{*}
\checkandfixthelayout
% ---

% --- 
% Espaçamentos entre linhas e parágrafos 
% --- 

% O tamanho do parágrafo é dado por:
\setlength{\parindent}{1.3cm}

% Controle do espaçamento entre um parágrafo e outro:
\setlength{\parskip}{0.2cm}  % tente também \onelineskip

% Espaçamento simples
\SingleSpacing
% ---


% ----
% Início do documento
% ----
\begin{document}

\selectlanguage{brazil}
% Retira espaço extra obsoleto entre as frases.
\frenchspacing 

% ----------------------------------------------------------
% ELEMENTOS PRÉ-TEXTUAIS
% ----------------------------------------------------------

% página de titulo principal (obrigatório)
\maketitle

% ----------------------------------------------------------
% ELEMENTOS TEXTUAIS
% ----------------------------------------------------------
\textual

% ----------------------------------------------------------
\section{Introdução}
% ---
O mercado de capitais é um termômetro sensível às condições sociais e econômicas de um país. As cotações, lucros e índices do mercado, como o IBOVESPA, refletem tanto o desempenho corporativo, como também a confiança dos investidores.

Este trabalho tem como foco a análise estatística descritiva da ação do Banco do Brasil, a mais antiga 
instituição financeira do país, ao longo de um período significativo de 20 anos (2004 a 2024), abrangendo 
os governos Lula, Dilma, Temer e Bolsonaro, e o retorno ao governo Lula ao final da série histórica.

A motivação para este estudo reside na possibilidade de investigar, por meio de ferramentas estatísticas, 
como variáveis-chave — Preço da ação, Lucro por ação e o índice IBOVESPA — se comportaram e se relacionaram em diferentes contextos políticos e econômicos. Questões como \textit{"O lucro da empresa apresentou tendência de crescimento constante?"} ou \textit{"Existe uma correlação visível entre o preço da ação e o desempenho do IBOVESPA?"} guiam esta investigação.

Para isso, serão utilizados conceitos fundamentais da estatística descritiva, como medidas de tendência central e dispersão, análise de frequência, construção de gráficos e estudo de correlação. O objetivo principal é realizar uma exploração deste conjunto de dados, apresentando os resultados de forma clara e discutindo possíveis interpretações para os padrões observados.

% ---
\section{Metodologia}
\subsection{Fonte dos Dados}
Os dados históricos de preços das ações do Banco do Brasil (BBAS3.SA) e do índice IBOVESPA (BVSP), foram obtidos através da biblioteca yfinance do Python, que fornece acesso aos dados do Yahoo Finance. 
O período selecionado foi de março de 2004 a dezembro de 2004, abrangendo 20 anos de dados trimestrais. Foi escolhida esta fonte, por ser amplamente utilizada pela comunidade financeira e por sua confiabilidade, frequência de atualização dos dados e manutenção do código da biblioteca.

% falta referência:
Os dados de lucro líquido trimestral do Banco do Brasil foram coletados manualmente a partir dos relatórios financeiros trimestrais publicados pela instituição. Estes documentos estão disponíveis no site de relações com investidores do Banco do Brasil e representam informações auditadas e oficialmente divulgadas ao mercado.

% ---
\subsection{Ferramentas e Softwares Utilizados}
Para a coleta, processamento e análise dos dados, foram utilizadas as seguintes ferramentas: Python 3.12 com as bibliotecas pandas, para manipulação, yfinance, usado para a coleta dos dados financeiros, streamlit, para visualização interativa e plotly, para os gráficos.

% ---
\subsection{Variáveis Utilizadas}
Foram utilizadas quatro variáveis para analisar o desempenho da ação BBAS.SA entre 2004 e 2024. A seleção dessas variáveis buscou capturar tanto aspectos específicos do ativo quanto o ambiente social no qual está inserido.

A variável \textbf{Preço da ação} apresenta o valor de mercado no fechamento de cada trimestre, servindo de indicador direto da percepção no mercado. Paralelamente, o \textbf{Lucro Líquido Trimestral} reflete o desempenho operacional e a saúde financeira da instituição.

Para contextualizar o desempenho do ativo em relação ao mercado, incluiu-se o \textbf{IBOVESPA}, principal índice de  referência do mercado acionário. Esta variável permite observar se os movimentos do ativo estão ou não alinhados com tendências gerais do mercado, ou apresentam comportamento distinto.

Por fim, a variável \textbf{Período de Governo} introduz uma dimensão qualitativa importante, categorizando cada trimestre conforme o governo federal vigente. Permitindo examinar se contextos políticos podem influenciar o desempenho do setor financeiro, em especial o Banco do Brasil, instituição financeira de economia mista, com controle estatal.

Essas variáveis, duas quantitativas que explicitam preços, uma quantitativa que expressa um score e uma qualitativa, proporciona uma base de investigação interessante para explorar fatores que podem influir no desempenho da mais antiga instituição financeira do país.

O Anexo~\ref{tabela-completa} apresenta a base de dados completa utilizada nesta analise.

% ---
\section{Análise Exploratória dos Dados}
Os dados coletados foram organizados em uma tabela única, totalizando 84 observações trimestrais, referentes ao período de 2004 a 2024. 
Cada registro contém as seguintes variáveis: índice de identificação, trimestre de referência, nome do ativo, (BBAS3.SA), preço da ação (em reais), lucro líquido trimestral (em bilhão de reais), IBOVESPA (pontos) e período de governo vigente.

\begin{figure}
	\centering
	\includegraphics[width=0.9\textwidth]{freqGov.png}
	\title{Distribuição de Frequência Absoluta dos Governos}
	\caption{Distribuição de Frequência - Período de Governo (2004-2024).}
	\label{fig:freq_gov}
\end{figure}

Optou-se por tratar os dados como população, uma vez que o conjunto abrange integralmente o período de interesse definido para este estudo, não havendo intenção de inferência para períodos externos a esta janela temporal.

No decorrer dos 83 trimestres, foram três presidentes eleitos e um governo interino. O presidente Lula, predominou no período, foram 36 trimestres, que compreendem, seu primeiro mandato (2003-2006), sua posterior reeleição (2007-2006) e seu terceiro mandato (2023-2024). A distribuição de frequência dos governos pode ser vista na Tabela~\ref{freq_gov} e também no Figura~\ref{fig:freq_gov}.

\begin{table}[h]
\IBGEtab{%
  \caption{Tabela de frequência absoluta e relativa dos Governos}%
  \label{freq_gov}
}{%
\begin{tabular}{c l r r}
 & Governo & Frequência Relativa & Frequência Absoluta \\
 \midrule \midrule
 0 & Governo Lula & 42.857143 & 36 \\
 1 & Governo Dilma & 28.571429 & 24 \\
 2 & Governo Bolsonaro & 19.047619 & 16 \\
 3 & Governo Temer & 9.523810 & 8 \\
\end{tabular}%
}{%
  \fonte{Produzido pelos autores.}%
  }
\end{table}

Usando os quartis, analisou-se a distribuição do lucro líquido trimestral, os resultados indicam que 25\% dos trimestres apresentam lucro inferior a 2,35 bilhões de reais, enquanto 75\% dos trimestres tiveram lucro abaixo de R\$ 4,16 bilhões, é possível observar um ponto bastante discrepante, no quarto trimestre de 2021 onde o lucro líquido da instituição foi superior a R\$ 10 bilhões (Figura~\ref{boxplot_lucro}).

\begin{figure}[h]
    \centering
    \includegraphics[width=0.9\textwidth]{boxplotLucro.png}
    \caption{Distribuição do Lucro Líquido por Quartis - Boxplot'}
    \label{boxplot_lucro}
\end{figure}

% ----------------------------------------------------------
\subsection{Características Gerais das Variáveis}
A análise inicial dos dados permite observar tendências e variações relevantes ao longo da série histórica. 
A variável \textit{Preço} da ação do Banco do Brasil (BBAS3.SA) apresentou uma trajetória geral de crescimento, com valor mínimo de R\$ 1,22 e máximo de R\$ 25,40, resultando em uma amplitude de R\$ 24,18 ao longo do período. 
A volatilidade dessa variável, medida pelo desvio padrão ($\sigma$), foi de 6,1459, enquanto a variância ($\sigma^2$) foi de 37,7723, refletindo flutuações consideráveis no preço da ação ao longo dos 20 anos 
analisados.

A Tabela~\ref{variancia} sintetiza as medidas de dispersão para as três variáveis quantitativas analisadas: Preço, Lucro e IBOVESPA. 
% ---
\begin{table}[h]
\IBGEtab{%
  \caption{Variância ($\sigma^2$) e desvio padrão ($\sigma$) no período de 2004 a 2014}%
  \label{variancia}
}{%
  \begin{tabular}{l r r}
  Variável & $\sigma^2$ & $\sigma$  \\
  \midrule \midrule
    Preço & 37,7723 & 6,1459 \\
	Lucro & 5,2904 & 2,3001 \\
	IBOVESPA & 920158777,6872 & 30334,1190 \\
  \bottomrule
\end{tabular}%
}{%
  \fonte{Produzido pelos autores.}%
  }
\end{table}
% ---

Além das medidas de dispersão, foram calculadas as principais medidas de tendência central, média ($\overline{x}$), mediana ($\tilde{x}$), moda ($M_o$), as quais oferecem uma visão consolidada do comportamento de cada uma das variáveis. Esses resultados são apresentados na Tabela~\ref{tendencia-central}.

% ---
\begin{table}[h]
\IBGEtab{%
  \caption{Medidas de tendência central no período de 2004 a 2014}%
  \label{tendencia-central}
}{%
  \begin{tabular}{l r r r}
  Variável & $\overline{x}$ & $\tilde{x}$ & $M_o$  \\
  \midrule \midrule
    Preço & 8,2031 & 5,6700 & {5,67} \\
	Lucro & 3,5215 & 2,9100 & {2,91; 7,45} \\
	IBOVESPA & 69976,6071 & 61243,0000 & --- \\
  \bottomrule
\end{tabular}%
}{%
  \fonte{Produzido pelos autores.}%
  }
\end{table}

% ---
\subsection{Lucro Líquido Trimestral vs IBOVESPA}
É possível visualizar mais rapidamente as variações considerando suas representações gráficas. 
A Figura~\ref{fig:preco-lucro-ibov} apresenta o gráfico de linhas com eixo duplo apresenta a evolução temporal comparativa entre o índice IBOVESPA (linha vermelha, eixo direito, escala de 0 a 140 mil pontos) e o Lucro Líquido trimestral (linha azul, eixo esquerdo, escala de 0 R\$10 Bi)

Observa-se uma tendência geral de crescimento para ambas as variáveis ao longo do período. O IBOVESPA apresenta crescimento mais acentuado a partir de 2016, com destaque para o período entre 2019 e 2021, quando atingiu seus valores máximos próximos a 130 mil pontos. Já o Lucro Líquido Trimestral apresenta maior volatilidade, com quedas acentuadas em 2015-2016 (chegando próximo a zero) e subsequente recuperação, alcançando valores superiores a 9 bilhões de reais nos períodos mais recentes.
Um ponto de inflexão notável ocorre em 2016, quando o lucro trimestral apresenta o menor valor registrado na série. Este desempenho pode estar relacionado ao contexto econômico e político do período.

\begin{figure}[h]
	\centering
	\includegraphics[width=0.9\textwidth]{ibovespaVSlucro.png}
	\caption{Gráficos de linhas comparando lucro líquido trimestral e IBOVESPA ao longo do tempo (2004-2024).}
	\label{fig:preco-lucro-ibov}
\end{figure}
% ---

\subsection{Lucro Médio Anual}
A Figura~\ref{fig:lucro-medio-anual} apresenta um histograma que consolida o lucro líquido médio anual do Banco do Brasil ao longo dos 20 anos analisados (2004-2024). Esta visualização permite identificar a evolução da rentabilidade da instituição em uma perspectiva temporal mais ampla.

Observa-se uma trajetória ascendente consistente no lucro médio anual, particularmente a partir de 2015. Os primeiros anos da série (2004-2008) apresentam valores mais modestos, com lucros médios entre 0,7 e 2,5 bilhões de reais. O período de 2009 a 2015 mostra crescimento gradual, com valores oscilando entre 2,5 e 4,5 bilhões de reais.

\begin{figure}[h]
	\centering
	\includegraphics[width=0.9\textwidth]{lucroMedioAno.png}
	\caption{Histograma do lucro médio anual do Banco do Brasil (2004-2024).}
	\label{fig:lucro-medio-anual}
\end{figure}

\subsection{Correlação}
A análise de correlação de Pearson (Tabela~\ref{matriz-correlacao} e Figura~\ref{fig:correlacao}) revelou diferentes graus de associação linear tanto a correlação Preço-Lucro ($\rho=0,714$), quanto a correlação Lucro-IBOVESPA ($\rho=0,724$) são consideradas positivas forte, o que indica que aumentos em uma são associadas ao aumento da outra. A correlação Preço-IBOVESPA ($\rho=0,908$) é positiva muito forte, evidenciando que a valorização da ação BBAS3.SA acompanha as tendências do mercado acionário. Essa correlação indica que, em média, um aumento de 1000 pontos no índice IBOVESPA está associado a uma variação de aproximadamente R\$ 0,18 no preço da ação BBAS3.SA, mantidas outras condições. 

O coeficiente de determinação ($R^2$) Preço-IBOVESPA é de 0,824, indicando que 82,4\% da variação do preço da ação pode ser explicada linearmente pelas variações do índice IBOVESPA. Este valor corresponde exatamente ao quadrado do coeficiente correlação de Pearson ($\rho^2=0,802^2$). O diagrama de dispersão (Figura~\ref{fig:preco-ibovespa}) confirma visualmente a correlação quase linear calculada.
Como sempre, correlação não necessariamente implica causalidade, outros fatores, tanto macro quanto específicos podem afetar diretamente o preço da ação.

\begin{figure}[h]
	\centering
	\includegraphics[width=0.5\textwidth]{relacaoPrecoIBOVESPA.png}
	\caption{Relação entre IBOVESPA e Preço da Ação.}
	\label{fig:preco-ibovespa}
\end{figure}
% ---

% ---
\begin{table}[h]
\IBGEtab{%
  \caption{Matriz de correlação de Pearson}%
  \label{matriz-correlacao}
}{%
  \begin{tabular}{l r r r}
   & Preço & Lucro & IBOVESPA  \\
  \midrule \midrule
    Preço & 1,000 & 0,714 & 0,908 \\
	Lucro & 0,714 & 1,000 & 0,724 \\
	IBOVESPA & 0,908 & 0,724 & 1,000 \\
  \bottomrule
\end{tabular}%
}{%
  \fonte{Produzido pelos autores.}%
  }
\end{table}

\begin{figure}[h]
	\centering
	\includegraphics[width=0.5\textwidth]{matrizCorrelacao.png}
	\caption{Matriz de Correlação de Variáveis.}
	\label{fig:correlacao}
\end{figure}
% ----------------------------------------------------------
% Finaliza a parte no bookmark do PDF
% para que se inicie o bookmark na raiz
% e adiciona espaço de parte no Sumário
% ----------------------------------------------------------
\bookmarksetup{startatroot}% 

% ---
% Conclusão (outro exemplo de capítulo sem numeração e presente no sumário)
% ---
\section{Conclusão}
% ---
Os resultados revelaram que os preços da ação apresentou trajetória consistente de valorização ao longo do período analisado, saindo de patamares inferiores a R\$ 2,00 em 2004 e atingindo valores superiores a R\$ 25,00 em 2024. A correlação muito forte entre preço e índices, demonstrando que 82,4\% da variação do preço da BBAS3.SA pode ser explicada linearmente pelo movimento do mercado. Esta forte associação confirma a sensibilidade do ativo às condições macroeconômicas gerais.

Por fim, o trabalho demonstrou a aplicabilidade dos conceitos estatísticos básicos - medidas de tendência central, dispersão, análise de correlação e representações gráficas - na compreensão de fenômenos econômicos complexos, atendendo ao objetivo de explorar o conjunto de dados e apresentar interpretações claras para os padrões observados.

% ----------------------------------------------------------
% ELEMENTOS PÓS-TEXTUAIS
% ----------------------------------------------------------
% ----------------------------------------------------------
\postextual % Inicia a parte pós-textual (sem numeração)
% ----------------------------------------------------------
\citeoption{abnt-full-initials=yes}
\nocite{aroussi,nield,morettin,investsite}
% ---
% Referências bibliográficas: Estrutura ABNTEX2 Completa
% ---

\bibliography{PCC_INE}
\newpage
% ---
% Anexos
% ---
\begin{anexosenv}
\chapter{Dados da Ação do Banco do Brasil (BBAS3) de 2004 a 2014}
\label{tabela-completa}

\begin{table}[h!]
\IBGEtab{%
  \caption{Dados da Ação do Banco do Brasil (BBAS3) de 2004 a 2014}%
  \label{tabela-csv-completo}
}{%
  \begin{tabular}{r l l r r r l}
  	oprule
  índice & Trimestre & Ativo & Preço & Lucro & IBOVESPA & Período Governo \\
  \midrule \midrule
  0 & Q1 01-2004 & BBAS3.SA & 1.26 & 0.61 & 22142 & Governo Lula \\
  1 & Q2 04-2004 & BBAS3.SA & 1.22 & 0.80 & 21149 & Governo Lula \\
  2 & Q3 07-2004 & BBAS3.SA & 1.37 & 0.83 & 23245 & Governo Lula \\
  3 & Q4 10-2004 & BBAS3.SA & 1.76 & 0.77 & 26196 & Governo Lula \\
  4 & Q1 01-2005 & BBAS3.SA & 1.60 & 0.96 & 26611 & Governo Lula \\
  5 & Q2 04-2005 & BBAS3.SA & 1.72 & 1.01 & 25051 & Governo Lula \\
  6 & Q3 07-2005 & BBAS3.SA & 2.35 & 1.44 & 31584 & Governo Lula \\
  7 & Q4 10-2005 & BBAS3.SA & 2.29 & 0.73 & 33456 & Governo Lula \\
  8 & Q1 01-2006 & BBAS3.SA & 3.01 & 2.34 & 37952 & Governo Lula \\
  9 & Q2 04-2006 & BBAS3.SA & 2.71 & 1.55 & 36631 & Governo Lula \\
  10 & Q3 07-2006 & BBAS3.SA & 2.57 & 0.90 & 36449 & Governo Lula \\
  11 & Q4 10-2006 & BBAS3.SA & 3.47 & 1.25 & 44474 & Governo Lula \\
  12 & Q1 01-2007 & BBAS3.SA & 3.61 & 0.00 & 45805 & Governo Lula \\
  13 & Q2 04-2007 & BBAS3.SA & 4.53 & 2.48 & 54392 & Governo Lula \\
  14 & Q3 07-2007 & BBAS3.SA & 5.02 & 1.36 & 60465 & Governo Lula \\
  15 & Q4 10-2007 & BBAS3.SA & 4.94 & 1.22 & 63886 & Governo Lula \\
  16 & Q1 01-2008 & BBAS3.SA & 3.76 & 2.35 & 60968 & Governo Lula \\
  17 & Q2 04-2008 & BBAS3.SA & 4.29 & 1.64 & 65018 & Governo Lula \\
  18 & Q3 07-2008 & BBAS3.SA & 3.73 & 1.87 & 49541 & Governo Lula \\
  19 & Q4 10-2008 & BBAS3.SA & 2.41 & 2.94 & 37550 & Governo Lula \\
  20 & Q1 01-2009 & BBAS3.SA & 2.76 & 1.67 & 40926 & Governo Lula \\
  21 & Q2 04-2009 & BBAS3.SA & 3.50 & 2.35 & 51465 & Governo Lula \\
  22 & Q3 07-2009 & BBAS3.SA & 5.24 & 1.98 & 61518 & Governo Lula \\
  23 & Q4 10-2009 & BBAS3.SA & 4.90 & 4.16 & 68588 & Governo Lula \\
  24 & Q1 01-2010 & BBAS3.SA & 5.17 & 2.35 & 70372 & Governo Lula \\
  25 & Q2 04-2010 & BBAS3.SA & 4.33 & 2.73 & 60936 & Governo Lula \\
  26 & Q3 07-2010 & BBAS3.SA & 5.73 & 2.62 & 69430 & Governo Lula \\
  27 & Q4 10-2010 & BBAS3.SA & 5.66 & 3.59 & 69305 & Governo Lula \\
  28 & Q1 01-2011 & BBAS3.SA & 5.42 & 2.93 & 68587 & Governo Dilma \\
  29 & Q2 04-2011 & BBAS3.SA & 5.21 & 3.33 & 62404 & Governo Dilma \\
  30 & Q3 07-2011 & BBAS3.SA & 4.72 & 2.89 & 52324 & Governo Dilma \\
  31 & Q4 10-2011 & BBAS3.SA & 4.58 & 4.98 & 56754 & Governo Dilma \\
  32 & Q1 01-2012 & BBAS3.SA & 5.10 & 2.55 & 64511 & Governo Dilma \\
  33 & Q2 04-2012 & BBAS3.SA & 3.90 & 3.00 & 54355 & Governo Dilma \\
  34 & Q3 07-2012 & BBAS3.SA & 5.04 & 2.80 & 59176 & Governo Dilma \\
  35 & Q4 10-2012 & BBAS3.SA & 5.30 & 3.54 & 60952 & Governo Dilma \\
  36 & Q1 01-2013 & BBAS3.SA & 5.68 & 2.58 & 56352 & Governo Dilma \\
  37 & Q2 04-2013 & BBAS3.SA & 4.73 & 7.45 & 47457 & Governo Dilma \\
  38 & Q3 07-2013 & BBAS3.SA & 5.82 & 2.69 & 52338 & Governo Dilma \\
  39 & Q4 10-2013 & BBAS3.SA & 5.58 & 2.74 & 51507 & Governo Dilma \\
  40 & Q1 01-2014 & BBAS3.SA & 5.33 & 2.77 & 50415 & Governo Dilma \\
  41 & Q2 04-2014 & BBAS3.SA & 5.91 & 2.79 & 53168 & Governo Dilma \\
  \bottomrule
\end{tabular}%
}{%
  \fonte{Produzido pelos autores.}%
  }
\end{table}
\begin{table}[h!]
\IBGEtab{%
  \caption{Continuação da Tabela~\ref{tabela-csv-completo}: Dados da Ação do Banco do Brasil (BBAS3) de 2004 a 2014}%
}{%
  \begin{tabular}{r l l r r r l}
  	oprule
  índice & Trimestre & Ativo & Preço & Lucro & IBOVESPA & Período Governo \\
  \midrule \midrule
  42 & Q3 07-2014 & BBAS3.SA & 6.09 & 2.74 & 54116 & Governo Dilma \\
  43 & Q4 10-2014 & BBAS3.SA & 5.82 & 4.15 & 50007 & Governo Dilma \\
  44 & Q1 01-2015 & BBAS3.SA & 5.72 & 5.68 & 51150 & Governo Dilma \\
  45 & Q2 04-2015 & BBAS3.SA & 6.25 & 2.93 & 53081 & Governo Dilma \\
  46 & Q3 07-2015 & BBAS3.SA & 4.01 & 3.01 & 45059 & Governo Dilma \\
  47 & Q4 10-2015 & BBAS3.SA & 3.99 & 6.37 & 43350 & Governo Dilma \\
  48 & Q1 01-2016 & BBAS3.SA & 5.40 & 2.31 & 50055 & Governo Dilma \\
  49 & Q2 04-2016 & BBAS3.SA & 4.77 & 2.44 & 51527 & Governo Dilma \\
  50 & Q3 07-2016 & BBAS3.SA & 6.41 & 2.22 & 58367 & Governo Dilma \\
  51 & Q4 10-2016 & BBAS3.SA & 7.95 & -0.67 & 60227 & Governo Dilma \\
  52 & Q1 01-2017 & BBAS3.SA & 9.58 & 2.39 & 64984 & Governo Temer \\
  53 & Q2 04-2017 & BBAS3.SA & 7.68 & 2.62 & 62900 & Governo Temer \\
  54 & Q3 07-2017 & BBAS3.SA & 10.09 & 2.81 & 74294 & Governo Temer \\
  55 & Q4 10-2017 & BBAS3.SA & 9.29 & 2.93 & 76402 & Governo Temer \\
  56 & Q1 01-2018 & BBAS3.SA & 12.07 & 2.73 & 85366 & Governo Temer \\
  57 & Q2 04-2018 & BBAS3.SA & 8.51 & 3.11 & 72763 & Governo Temer \\
  58 & Q3 07-2018 & BBAS3.SA & 8.85 & 3.09 & 79342 & Governo Temer \\
  59 & Q4 10-2018 & BBAS3.SA & 14.29 & 6.12 & 87887 & Governo Temer \\
  60 & Q1 01-2019 & BBAS3.SA & 15.19 & 3.92 & 95415 & Governo Bolsonaro \\
  61 & Q2 04-2019 & BBAS3.SA & 17.01 & 4.20 & 100967 & Governo Bolsonaro \\
  62 & Q3 07-2019 & BBAS3.SA & 14.56 & 4.16 & 104745 & Governo Bolsonaro \\
  63 & Q4 10-2019 & BBAS3.SA & 17.10 & 8.70 & 115964 & Governo Bolsonaro \\
  64 & Q1 01-2020 & BBAS3.SA & 9.16 & 3.19 & 73020 & Governo Bolsonaro \\
  65 & Q2 04-2020 & BBAS3.SA & 10.55 & 3.16 & 95056 & Governo Bolsonaro \\
  66 & Q3 07-2020 & BBAS3.SA & 9.89 & 3.04 & 94603 & Governo Bolsonaro \\
  67 & Q4 10-2020 & BBAS3.SA & 13.07 & 4.15 & 119306 & Governo Bolsonaro \\
  68 & Q1 01-2021 & BBAS3.SA & 10.46 & 4.16 & 116634 & Governo Bolsonaro \\
  69 & Q2 04-2021 & BBAS3.SA & 11.23 & 5.57 & 126802 & Governo Bolsonaro \\
  70 & Q3 07-2021 & BBAS3.SA & 10.29 & 4.54 & 110979 & Governo Bolsonaro \\
  71 & Q4 10-2021 & BBAS3.SA & 10.47 & 10.64 & 104822 & Governo Bolsonaro \\
  72 & Q1 01-2022 & BBAS3.SA & 12.97 & 4.57 & 119999 & Governo Bolsonaro \\
  73 & Q2 04-2022 & BBAS3.SA & 12.80 & 8.20 & 98542 & Governo Bolsonaro \\
  74 & Q3 07-2022 & BBAS3.SA & 15.15 & 7.51 & 110037 & Governo Bolsonaro \\
  75 & Q4 10-2022 & BBAS3.SA & 14.13 & 7.35 & 110031 & Governo Bolsonaro \\
  76 & Q1 01-2023 & BBAS3.SA & 16.38 & 6.69 & 101882 & Governo Lula \\
  77 & Q2 04-2023 & BBAS3.SA & 21.20 & 8.36 & 118087 & Governo Lula \\
  78 & Q3 07-2023 & BBAS3.SA & 20.74 & 8.04 & 116565 & Governo Lula \\
  79 & Q4 10-2023 & BBAS3.SA & 24.51 & 6.77 & 134185 & Governo Lula \\
  80 & Q1 01-2024 & BBAS3.SA & 25.23 & 8.03 & 128106 & Governo Lula \\
  81 & Q2 04-2024 & BBAS3.SA & 24.39 & 8.70 & 123907 & Governo Lula \\
  82 & Q3 07-2024 & BBAS3.SA & 25.40 & 5.27 & 131816 & Governo Lula \\
  83 & Q4 10-2024 & BBAS3.SA & 23.18 & 4.37 & 120283 & Governo Lula \\
  \bottomrule
\end{tabular}%
}{%
  \fonte{Produzido pelos autores.}
  }
\end{table}

\end{anexosenv}

\end{document}